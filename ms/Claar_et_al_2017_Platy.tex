\documentclass[]{article}
\usepackage{lmodern}
\usepackage{amssymb,amsmath}
\usepackage{ifxetex,ifluatex}
\usepackage{fixltx2e} % provides \textsubscript
\ifnum 0\ifxetex 1\fi\ifluatex 1\fi=0 % if pdftex
  \usepackage[T1]{fontenc}
  \usepackage[utf8]{inputenc}
\else % if luatex or xelatex
  \ifxetex
    \usepackage{mathspec}
  \else
    \usepackage{fontspec}
  \fi
  \defaultfontfeatures{Ligatures=TeX,Scale=MatchLowercase}
\fi
% use upquote if available, for straight quotes in verbatim environments
\IfFileExists{upquote.sty}{\usepackage{upquote}}{}
% use microtype if available
\IfFileExists{microtype.sty}{%
\usepackage{microtype}
\UseMicrotypeSet[protrusion]{basicmath} % disable protrusion for tt fonts
}{}
\usepackage[margin=1in]{geometry}
\usepackage{hyperref}
\hypersetup{unicode=true,
            pdftitle={KI Platygyra Manuscript},
            pdfauthor={Danielle C Claar, Kristina L Tietjen, Ruth D Gates, Julia K Baum},
            pdfborder={0 0 0},
            breaklinks=true}
\urlstyle{same}  % don't use monospace font for urls
\usepackage{graphicx,grffile}
\makeatletter
\def\maxwidth{\ifdim\Gin@nat@width>\linewidth\linewidth\else\Gin@nat@width\fi}
\def\maxheight{\ifdim\Gin@nat@height>\textheight\textheight\else\Gin@nat@height\fi}
\makeatother
% Scale images if necessary, so that they will not overflow the page
% margins by default, and it is still possible to overwrite the defaults
% using explicit options in \includegraphics[width, height, ...]{}
\setkeys{Gin}{width=\maxwidth,height=\maxheight,keepaspectratio}
\IfFileExists{parskip.sty}{%
\usepackage{parskip}
}{% else
\setlength{\parindent}{0pt}
\setlength{\parskip}{6pt plus 2pt minus 1pt}
}
\setlength{\emergencystretch}{3em}  % prevent overfull lines
\providecommand{\tightlist}{%
  \setlength{\itemsep}{0pt}\setlength{\parskip}{0pt}}
\setcounter{secnumdepth}{5}
% Redefines (sub)paragraphs to behave more like sections
\ifx\paragraph\undefined\else
\let\oldparagraph\paragraph
\renewcommand{\paragraph}[1]{\oldparagraph{#1}\mbox{}}
\fi
\ifx\subparagraph\undefined\else
\let\oldsubparagraph\subparagraph
\renewcommand{\subparagraph}[1]{\oldsubparagraph{#1}\mbox{}}
\fi

%%% Use protect on footnotes to avoid problems with footnotes in titles
\let\rmarkdownfootnote\footnote%
\def\footnote{\protect\rmarkdownfootnote}

%%% Change title format to be more compact
\usepackage{titling}

% Create subtitle command for use in maketitle
\newcommand{\subtitle}[1]{
  \posttitle{
    \begin{center}\large#1\end{center}
    }
}

\setlength{\droptitle}{-2em}
  \title{KI Platygyra Manuscript}
  \pretitle{\vspace{\droptitle}\centering\huge}
  \posttitle{\par}
  \author{Danielle C Claar, Kristina L Tietjen, Ruth D Gates, Julia K Baum}
  \preauthor{\centering\large\emph}
  \postauthor{\par}
  \predate{\centering\large\emph}
  \postdate{\par}
  \date{21 September, 2016}

\begin{document}
\maketitle

\begin{verbatim}
## Loading required package: qdapDictionaries
\end{verbatim}

\begin{verbatim}
## Loading required package: qdapRegex
\end{verbatim}

\begin{verbatim}
## Loading required package: qdapTools
\end{verbatim}

\begin{verbatim}
## Loading required package: RColorBrewer
\end{verbatim}

\begin{verbatim}
## 
## Attaching package: 'qdap'
\end{verbatim}

\begin{verbatim}
## The following object is masked from 'package:base':
## 
##     Filter
\end{verbatim}

\begin{verbatim}
## [1] NA
\end{verbatim}

\emph{Titles should not exceed 75 characters (including spaces) for
Articles. Titles should not include numbers, acronyms, abbreviations or
punctuation. They should include sufficient detail for indexing purposes
but be general enough for readers outside the field to appreciate what
the paper is about.}

\emph{Letters no more than 4 pages, of Nature. An uninterrupted page of
text contains about 1,300 words.}

\emph{Letters are short reports of original research focused on an
outstanding finding whose importance means that it will be of interest
to scientists in other fields. As a guideline they allow up to 30
references. They begin with a fully referenced paragraph, of about 200
words, (certainly no more than 300 words) aimed at readers in other
disciplines. This starts with a 2--3 sentence basic introduction to the
field; followed by a one-sentence statement of the main conclusions
starting `Here we show' or equivalent phrase; and 2--3 sentences putting
the main findings into general context. See the information sheet `How
to construct a Nature summary paragraph' for an annotated example. The
rest of the text is typically about 1,500 words long (not including
Methods, summary paragraph or other sections). Letters have 3 or 4 small
display items.}

\section{Summary}\label{summary}

Coral reefs, which already live on the edge of their thermal tolerance
(CITE), are under acute threat from El Nino-associated ocean warming
(CITE). The 2015/16 El Nino is currently the worst on record in terms of
severity and longevity (1), yet despite massive coral mortality, some
corals show resilience to this extreme event. El Nino warming is
intensifying (2, 3, 4), threatening coral reefs and endangering the
persistence of vital ecosystem services, threatening the food security
and coastline protection of coastal communities worldwide (CITE). Coral
resilience is related to many factors (CITE), including the structure
and flexibility of their internal symbiotic communities (CITE,CITE).
However, the mechanisms of coral resilience to extended pulse stress
events are still poorly understood. Here, using Illumina MiSeq amplicon
sequencing, we show that, contrary to current opinion (CITE), symbionts
present in miniscule abundances (\textless{}2\%) are indeed important
for coral recovery. Additionally, we show that the ability of a coral to
house a diverse suite of symbionts (\emph{Symbiodinium}) is driven by
the identity of the dominant \emph{Symbiodinium} type. Furthermore, we
show that corals have the ability to regain symbionts during an extended
stress event, providing hope for the future of coral resilience.

\section{Main Text}\label{main-text}

\emph{The rest of the text is typically about 1,500 words long (not
including Methods, summary paragraph or other sections). Letters have 3
or 4 small display items.}

\emph{Little bit more background} Ocean ecosystems worldwide are
threatened by climate change-induced increases in seawater temperatures
(CITE, CITE, CITE). Pulse warming events such as El Nino amplify these
threats, causing massive losses of coral cover (e.g.~17\% of coral reefs
during the 1997/98 El Nino) (CITE,CITE). El Niño, the positive phase of
the ENSO (El Niño Southern Oscillation), is a natural climatic event
that occurs when surface waters heat up in the equatorial Pacific, that
causes catastrophic effects on reef ecosystems by disrupting coral
symbioses (i.e.~coral bleaching). El nino warming threatens coral reefs
by disrupting the dynamic symbiosis between coral and their internal
symbiotic algae (Symbiodinium). This symbiosis is the foundation of reef
ecosystems, and a critical element of reef resilience (van oppen and
gates 2006). Corals host a diverse community of Symbiodinium, ranging
along a continuum from `selfish opportunistic symbionts' (e.g.~some
clade D Symbiodinium) which are better suited to sustained environmental
stress than others, to `intimately evolved symbionts' which provide
exceptional amounts of nutrition to their coral host (5). Thus, although
these relationships have developed over evolutionary time, the
resilience of the coral symbiome is constantly shaped by dynamic
coral-symbiont interactions (6). The 2015-2016 El Niño is the first
major global event since 1997-1998, and has been declared the third
global coral bleaching event by NOAA (1). Kiritimati Atoll (Christmas
Island), located in the Central Equatorial Pacific, is at the epicenter
of this extreme El Niño event. Thermal anomalies were severe on
Kiritimati, reaching an unprecedented (cite o h-g) XX number of DHW over
a XX month long bleaching event, demolishing most of the reef
(\textbf{???}).

\emph{Briefest methods plus results summary} Here we assess
\emph{briefest methods section ever} Here, we tagged and sampled the
same corals before, during, and immediately after the el nino event, on
Kiritimati (something about Kiritimati). We used Illumina sequencing to
evaluate changes in symbiodinium community structure coincident with the
2015-2016 major el nino event. The goal was to understand\ldots{}why the
hell these corals survived 10 months of extreme heat stress, and
actually got better in the middle of it. Example of when this high-risk
ecological opportunity (7) actually pays off\ldots{}

\section{\texorpdfstring{\emph{Tiny abundances are
important}}{Tiny abundances are important}}\label{tiny-abundances-are-important}

\emph{We used to think that bleaching might be good - ABH says that
corals bleach in order to expel suboptimal Symbiodinium types in
exchange for optimal symbionts during the new conditions (8,
Baker:2001bf, 9)} \emph{We do know that corals house background
symbionts in low abundances (10, all the recent ngs studies\ldots{}),
but these relationships have been described as unstable (11, more
cites?). }Switching and shuffling (12)* \emph{And we do know that some
symbio are ``better'' than others} \emph{And then we said that bleaching
is definitely bad} \emph{But at least we do know that it allows changes
to occur in the Symbiodinium community structure} \emph{Figure of
example sequence abundances, superimposed on coral 99 images}

\section{\texorpdfstring{\emph{Dominant symbiont drives sym
div}}{Dominant symbiont drives sym div}}\label{dominant-symbiont-drives-sym-div}

\emph{Regardless it is likely that symbiodinium community composition is
important for resilience, corals that host flexible symbioses may be
more sensitive to environmental changes (13)} \emph{Taxa-specific
bleaching is a thing} \emph{Figure about diversity}

\section{\texorpdfstring{\emph{Corals can regain symbionts,
whoa!}}{Corals can regain symbionts, whoa!}}\label{corals-can-regain-symbionts-whoa}

\emph{When we've seen resilience/recovery before, corals have only been
demonstrated to recover if the stress goes away first} Previously,
corals have been shown to recover from bleaching only after the external
stress (e.g.~warming) has subsided. \emph{implying that longer and more
frequent stressors spell disaster for reefs worldwide}

\section{\texorpdfstring{\emph{In a sea of destruction, a glimmer of
hope}}{In a sea of destruction, a glimmer of hope}}\label{in-a-sea-of-destruction-a-glimmer-of-hope}

\emph{we have shown\ldots{}.why this is important} (Methods like NGS are
really important for understanding these changes in symbiont diversity,
as well as for seeing those low aboundance symbionts) (What does their
recovery tell us about the future of coral reefs?)\ldots{} Why are
Platys so excellent and what does that tell us about when coral
resilience is threatened by extreme climatic events? Elucidating the
mechanisms underlying changes in coral-symbiont interactions is
essential to understanding the ability of the coral symbiome to adapt to
the multiple stressors they now face.

\section{Methods}\label{methods}

\emph{The Methods section should be written as concisely as possible but
should contain all elements necessary to allow interpretation and
replication of the results. As a guideline, Methods sections typically
do not exceed 3,000 words. Detailed descriptions of methods already
published should be avoided; a reference number can be provided to save
space, with any new addition or variation stated. The Methods section
should be subdivided by short bold headings referring to methods used
and we encourage the inclusion of specific subsections for statistics,
reagents and animal models. If further references are included in this
section, the numbering should continue from the end of the last
reference number in the rest of the paper and the list should accompany
the additional Methods at the end of the paper. The Methods section
cannot contain figures or tables (essential display items should be
included in the Extended Data).}

\subsection{Field Information}\label{field-information}

Kiritimati basics - located in the Central Equatorial Pacific, smack dab
in the middle of the Nino 3.4 region (used to quantify el nino presence
and strength), human disturbance gradient, bleaching event there (cite
bleaching paper here) Tagging corals and collecting samples - transects,
tagging corals, photoing corals, sampling corals, processing samples,
storage in Guanidinium Taxa sampled - platy, favites, favia, etc. \#\#
Pre-processing and sequencing DNA Extraction - extraction protocol ITS2
region - it's annoying, but it's the best we've got right now PCR and
Cleanup - Amy's method of PCR and cleanup Library Prep - Amy's method of
library prep, include Illumina Sequencing information (barcodes, etc)
\#\# Post Processing Sequence QC - boku, then merge with illumina utils,
max mismatch=3 Sequence clustering - denovo clustering using UCLUST in
QIIME, then compare to reference database to assign taxonomy Statistical
Analysis - alpha diversity of sequence reads, co-occurence?, beta
diversity?

\section{References}\label{references}

\hypertarget{refs}{}
\hypertarget{ref-Eakin:2016vf}{}
1. Eakin, C. M. \emph{et al.} Global Coral Bleaching 2014-2017: Status
and an Appeal for Observations. \emph{Reef Encounter} \textbf{31,}
20--26 (2016).

\hypertarget{ref-Trenberth:1997vw}{}
2. Trenberth, K. E. \& Hoar, T. J. El Niño and climate change.
\emph{Geophysical Research Letters} \textbf{24,} 3057--3060 (1997).

\hypertarget{ref-Cobb:2013fe}{}
3. Cobb, K. M. \emph{et al.} Highly variable El Niño-Southern
Oscillation throughout the Holocene. \emph{Science} \textbf{339,} 67--70
(2013).

\hypertarget{ref-Cai:2014ky}{}
4. Cai, W. \emph{et al.} Increasing frequency of extreme El Nino events
due to greenhouse warming. \emph{Nature Climate Change} \textbf{4,}
111--116 (2014).

\hypertarget{ref-Lesser:2013tu}{}
5. Lesser, M. P., Stat, M. \& Gates, R. D. The endosymbiotic
dinoflagellates (\emph{Symbiodinium} sp.) of corals are parasites and
mutualists. \emph{Coral Reefs} 1--9 (2013).

\hypertarget{ref-Stat:2006ww}{}
6. Stat, M., Carter, D. \& Hoegh-Guldberg, O. The evolutionary history
of \emph{Symbiodinium} and scleractinian hosts --- symbiosis, diversity,
and the effect of climate change. \emph{Perspectives in Plant Ecology}
\textbf{8,} 23--43 (2006).

\hypertarget{ref-Baker:2001bf}{}
7. Baker, A. C. Ecosystems: Reef corals bleach to survive change.
\emph{Nature} \textbf{411,} 765--766 (2001).

\hypertarget{ref-Buddemeier:1993bb}{}
8. Buddemeier, R. W. \& Fautin, D. G. Coral bleaching as an adaptive
mechanism. \emph{Bioscience} \textbf{43,} 320--326 (1993).

\hypertarget{ref-Buddemeier:2004vj}{}
9. Buddemeier, R. W., Baker, A. C., Fautin, D. G. \& Jacobs, J. R. in
(Coral health; \ldots{}, 2004).

\hypertarget{ref-Correa:2009jy}{}
10. Correa, A. M. S., McDonald, M. D. \& Baker, A. C. Development of
clade-specific \emph{Symbiodinium} primers for quantitative PCR (qPCR)
and their application to detecting clade D symbionts in Caribbean
corals. \emph{Marine biology} \textbf{156,} 2403--2411 (2009).

\hypertarget{ref-Coffroth:2010ju}{}
11. Coffroth, M. A., Poland, D. M., Petrou, E. L., Brazeau, D. A. \&
Holmberg, J. C. Environmental symbiont acquisition may not be the
solution to warming seas for reef-building corals. \emph{PLoS ONE}
\textbf{5,} e13258 (2010).

\hypertarget{ref-Baker:2003uo}{}
12. Baker, A. C. Flexibility and specificity in coral-algal symbiosis:
diversity, ecology, and biogeography of Symbiodinium. \emph{Annual
Review of Ecology, Evolution, and Systematics} 661--689 (2003).

\hypertarget{ref-Putnam:2012bn}{}
13. Putnam, H. M., Stat, M., Pochon, X. \& Gates, R. D. Endosymbiotic
flexibility associates with environmental sensitivity in scleractinian
corals. \emph{Proceedings of the Royal Society B: Biological Sciences}
\textbf{279,} 4352--4361 (2012).


\end{document}
